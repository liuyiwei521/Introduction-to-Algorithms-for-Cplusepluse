
%%%%%%%%%%%%%%%%%%%%%%%%%%%%%%%%%%%%%%%%%%%%%%%%%%%%%%%%%%%%%%%%%%%%%%%%%%%%%%%%%%%%%%%%%%
%  A rigidity theorem for complete noncompact manifolds with harmonic curvature.tex       %
%                          2010.08.17                                                     %
%%%%%%%%%%%%%%%%%%%%%%%%%%%%%%%%%%%%%%%%%%%%%%%%%%%%%%%%%%%%%%%%%%%%%%%%%%%%%%%%%%%%%%%%%%

\documentclass[9pt, a4paper,eqno]{article}
\usepackage{caption}
\usepackage{algorithm}
\usepackage{algpseudocode}
\usepackage{algorithmicx}
%\usepackage{algorithmic}
%\documentclass[leqno,11pt]{article}
\usepackage{CJKutf8}
\usepackage{amsfonts, amsmath, amssymb, amscd}
\usepackage{latexsym}
\usepackage{graphicx}
\usepackage{subfigure}
\usepackage{amsthm}
\usepackage{fancyhdr}
\usepackage{times}
\usepackage{mathptmx}
\usepackage{bm}
\newcommand{\figcaption}{\def\@captype{figure}\caption}
\newcommand{\tabcaption}{\def\@captype{table}\caption}

\renewcommand{\baselinestretch}{1.1}
\usepackage{geometry}\geometry{left=2.6cm,right=2.5cm,top=2.5cm,bottom=2.5cm}
\pagestyle{plain}

\theoremstyle{plain}
%\newtheorem{theorem}{Theorem}[section]
%\newtheorem{lemma}[theorem]{Lemma}
%\newtheorem{proposition}[theorem]{Proposition} %原来的
%\newtheorem{corollary}[theorem]{Corollary}
%\newtheorem{problem}[theorem]{Problem}

\newtheorem{theorem}{Theorem}[section]
\newtheorem{lemma}{Lemma}[section]
\newtheorem{proposition}{Proposition}[section]
\newtheorem{corollary}{Corollary}[section]
\newtheorem{remark}{Remark}[section]
\newtheorem{definition}{Definition}[section]
\newtheorem{condition}{Condition}[section]
\newtheorem{example}{Example}[section]
\newtheorem{conclusion}{Conclusion}[section]
%\newtheorem{algorithm}{Algorithm}[section]
\newtheorem{assumption}{Assumption}[section]
\renewenvironment{proof} {\par{\it Proof.} \ignorespaces} {\par\medskip}

\renewcommand{\theequation}{\arabic{section}.\arabic{equation}} %方程显示(2.1)表示第二节第一个方程

%\renewcommand{\footnoterule}{\noindent\rule{5pc}{0.25pt}\vspace*{6pt}}
%\renewcommand{\thefootnote}{\fnsymbol{footnote}}
%
%\theoremstyle{definition}
%\newtheorem{definition}[theorem]{Definition}
%\newtheorem{example}[theorem]{Example}
%\newtheorem{remark}[theorem]{Remark}

%\renewcommand{\thelemma}{\arabic{section}.\arabic{lemma}}
%\renewcommand\thetable{\thesection.\arabic{table}}

\newcommand{\grad}{\operatorname{grad}}
\newcommand{\tr}{\operatorname{tr}}
\newcommand{\vol}{\operatorname{vol}}
\newcommand{\Ric}{\operatorname{Ric}}
\newcommand{\Riem}{\operatorname{Riem}}
\newcommand{\me}{\mathrm{e}}
\newcommand{\mi}{\mathrm{i}}
\newcommand{\dif}{\mathrm{d}}
\newcommand{\hei}{\CJKfamily{hei}}
\newcommand{\song}{\CJKfamily{song}}

%%%%%%%%%%%%%%  TEXT  %%%%%%%%%%%%%%%%%%%%%%%%%%%%%%%%%%%%%%%%%%%%%%%%%%%%%%%%%
\begin{document}
\begin{CJK}{UTF8}{gkai}
\title{\LARGE
算法分析 \footnotetext{相关作者:\\
E-mail:jiangxizhengzhirun@163.com (郑智润)}
%\footnotetext{Supported by the National Natural Science Foundation
%of China (No. 10971203, 11101384, 11271340); Specialized Research
%Fund for the Doctoral Program of Higher Education (No.
%20094101110006). }
 }
\author {郑智润  \\
\small{湘潭大学数学与计算科学学院,湘潭, 411100 }  \\
}

\date{}

\maketitle \thispagestyle{empty} \large{
%\hspace*{-0.5cm}{\bf{摘要}}:
}

%========================================================================
%%%% Start %%%%%%
2.8~~~~假设我们需要生成前$N$个整数的一个随机排列.例如,$\left\lbrace 4,3,1,5,2  \right\rbrace$和
$\left\lbrace 3,1,4,2,5  \right\rbrace$就是合法的排列,但$\left\lbrace 5,4,1,2,1  \right\rbrace$
则不是,因为数1出现两次而数3却没有出现.这个程序常常用于模拟一些算法.假设存在一个随机数生成器$r$,它有方法$randInt(i,j)$,
该方法以相同的概率生成$i$和$j$之间的整数.下面是3个算法:
\begin{enumerate}
\item 如下填入从$a[0]$到$a[n-1]$的数组$a$.为了填入$a[i]$,生成随机数直到它不同于已经生成的$a[0],a[1],\cdots,a[i-1]$时再将
其填入$a[i]$.时间复杂度为$O(N^3)$.
\item 同算法1,但要保存一个附加的数组,称之为$used$数组.当一个随机数$ran$最初被放入数组$a$的时候,置$used[ran]=true$.这就是说,
当用一个随机数填入$a[i]$时,可以用一步来测试是否该随机数已经被使用,而不是像第一个算法那样(可能)用$i$步测试.时间复杂度为$O(N^2)$.
\item 填写该数组使得$a[i] = i+1$,然后
$$
for (i ~=~ 1;~ i~ <~ n;~ ++i)
~~~~~~~~~~~
		swap(a[i],a[randInt(0,i)]);
$$
时间复杂度为$O(N)$.
\end{enumerate}
答案: 代码 exercise2.8.cpp

                                                                                                                                                                                               2.13~~~~计算$$f(x) = \sum_{i=0}^{N}a_ix^i.$$
\begin{enumerate}
\item 采用$x^n = x \times x \times \cdots \times x$方式计算$x$的$n$次幂.求幂运算的时间复杂度为$O(N)$,求上述多项式的时间复杂度为$O(N^2)$.
\item 采用递归算法使求幂的时间复杂度降为$O(logN)$.$N \leq 1$是这种递归的基准情形.否则,若$N$是偶数,我们
有$x^N = x^{N/2} \cdot x^{N/2}$;如果$N$是奇数,则$x^N = x^{(N-1)/2}\cdot x^{(N-1)/2}\cdot x$.求上述多项式的时间复杂度为
$O(NlogN)$.
\item 采用Horner法则计算上述多项式时间复杂度为$O(N)$.$$\sum_{i=0}^{N}a_ix^i = ((((a_N) \times x + a_{N-1}) \times x + a_{N-2})\cdots) \times x + a_0.$$
\end{enumerate}

答案: 代码 exercise2.13.cpp

2.15~~~~给出一个有效的算法来确定在整数$A_1 < A_2 < A_3 < \cdots < A_N$的数组中是否在整数$i$使得$A_i = i$.你的算法运行时间是多少?

答案:采用二分策略进行查找,时间复杂度为$O(logN)$,代码 exercise2.15.cpp
\newpage
2.17~~~~给出有效的算法(及其运行时间分析)来:
\begin{enumerate}
\item 找出最小子序列和.
\item 找出最小正子序列和.
\item 找出最大序列乘积.
\end{enumerate}

答案: 

1.exercise2.17a.cpp给出三种算法分别为暴力搜索$O(N^2)$,递归分治策略$O(NlogN)$,联机算法$O(N)$.

2.exercise2.17b.cpp给出三种算法分别为暴力搜索$O(N^2)$,$O(NlogN)$.

3.exercise2.18c.cpp给出两种算法分别为暴力搜索$O(N^2)$,联机算法$O(N)$.

2.18~~~~数值分析中一个重要的问题是对某个任意的函数$f$找出方程$f(x)=0$的一个解.如果该函数是连续的并有两个点$low$和
$high$使得$f(low)$和$f(high)$符号相反,那么在$low$和$high$之间必然存在一个根,并且这个根可以通过折半查找求得.写出一个
函数,以$f,low,high$为参数,并且解出一个零点.为保证能够正常终止,你必须做什么?

答案:代码 exercise2.18.cpp

2.19~~~~正文中最大相连子序列和算法均不给出具体序列的任何指示.对这些算法进行修改,使得它们以单个对象的形式返回最大子序列的值以及
具体序列的那些相应的下标.

答案:代码 exercise2.19.cpp



















%=========================================================================
%%%% END %%%%
\end{CJK}
\end{document}
